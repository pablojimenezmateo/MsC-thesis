\chapter{State of the art}

This chapter gives an overview of previous work, and how this thesis fits within those studies.

\section{Smart cities}

In the last few years, smart cities have been a hot topic in research \cite{caragliu_bo_nijkamp_2011} given the quantity of available data provided by the internet of things \cite{zanella_bui_castellani_vangelista_zorzi_2014} and big data \cite{townsend_2013} and the newest infrastructures that allow cities to track activities in real time. This leds to the use of this data to improve the quality of life of the citizen, employment\cite{shapiro_2005} and also saving on unused services when they are not necessary.

Smart cities is a general idea of which smart transport is only a small part, but that is the part that this thesis is focusing on.

\section{Smart transport}

Smart transport is a topic that only recently has started to be researched. \cite{lenior_janssen_neerincx_schreibers_2006} defines Smart transport as \glqq adequate human–system symbiosis to realize effective, efficient and human-friendly transport of goods and information.\grqq and that is exactly what this thesis is focusing on, a protocol that will make transportation effective and efficient while being human-friendly.

As seen in this review \cite{perez_carrillo_montoya-torres_2014} the criteria to find the best alternatives is very varied, it can be based on  social, economic, environmental or even land usage. Our protocol is focused on minimize the network congestion and thus minimizing the total added time of travel times.

Our application helps analysts to test different routing algorithms in a distributed way so, and mix different algorithms within the same simulation. I believe that this tool will be really useful for anyone that wants an initial test since its usage and output are so easily learned.





